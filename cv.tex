\documentclass[12pt]{article}
\usepackage{graphicx, geometry, enumerate, float}
\usepackage{bibtopic}
\geometry{left=.5in,right=.5in,top=.5in,bottom=.5in,headsep=.1in}

\usepackage{helvet}
\renewcommand\familydefault{\sfdefault}

\usepackage{fancyhdr}

\pagestyle{fancy}
\lhead{}
\rhead{\footnotesize \today}
\chead{}
\cfoot{}
\lfoot{\footnotesize Brian S Caffo, Professor, Department of Biostatistics, Johns Hopkins University}
\rfoot{\thepage}
\renewcommand{\headrulewidth}{0pt} 
\renewcommand{\footrulewidth}{0.4pt}

\setlength{\parindent}{0.0in}

\usepackage{titlesec}
\titleformat{\section}{\titlerule \vspace{.8ex} \large\bf}{\thesection}{1ex}{}[]
\titleformat{\subsection}{\bf }{\thesection.\thesubsection}{1ex}{}[]


\begin{document}
%%the title
{ \vspace{-.5in}
\begin{center}
\large
\bf Curriculum Vitae \\
Brian S Caffo \\
%VERSION%
\end{center}
}

\section*{Personal information}
Department of Biostatistics\\
Johns Hopkins Bloomberg School of Public Health \\ 
615 North Wolfe Street, Baltimore, MD, 21205 \\ \\
Email:bcaffo@gmail.com \\
Homepages: www.bcaffo.com and www.smart-stats.org \\
github uid: bcaffo\\
Pronouns: he/him/his \\

\section*{Summary}
Brian Caffo, PhD is a professor in the Department of Biostatistics with a secondary appointment in the Department of Biomedical Engineering at Johns Hopkins University. 
He graduated from the University of Florida Department of Statistics in 2001. He has worked in statistical computing, statistical modeling, computational statistics, multivariate and decomposition methods and statistics in neuroimaging and neuroscience. He led teams that won the ADHD 200 prediction competition and placed 12th in the large Heritage Health competition. He co-directs the SMART statistical group. With other faculty at JHU, he created and co-directs the Coursera Data Science Specialization, a 10 course specialization on statistical data analysis. He has co-directs the JHU Data Science Lab, a group dedicated to open educational innovation and data science. He is the former director of the Biostatistics graduate programs and admissions committees. He is currently the co-director of the Johns Hopkins High Performance Computing Exchange and president-elect of the Bloomberg School of Public Health faculty senate.

\section*{Education and training}
\begin{description}
\item[\textnormal{2006}] NIH K25 training grant ``A mentored training program in imaging science'' emphasizing research and coursework in medical imaging
\item[\textnormal{2001}] PhD in statistics from the University of Florida Department of Statistics under Professor James Booth; thesis title ``Candidate sampling schemes and some important applications''
\item[\textnormal{1998}] MS in statistics from the University of Florida Department of Statistics
\item[\textnormal{1995}] BS in mathematics and statistics from the University of Florida's Departments of Mathematics and Statistics.
\end{description}


\section*{Professional experience}
\subsection*{Official appointments}
\begin{description}
\item[\textnormal{2019} - ] Secondary appointment, Department of Biomedical Engineering, Johns Hopkins University
\item[\textnormal{2019} - ] Co-director, Johns Hopkins High Performance Computing Exchange (JHPCE)
\item[\textnormal{2016} - ] Faculty member, Kavli Neuroscience Discovery Institute 
\item[\textnormal{2017} - ] Faculty member, Malone Center for Engineering and Healthcare
%\item[\textnormal{2016} - ] Full professor, Department of Biomedical Engineering, Johns Hopkins University
\item[\textnormal{2014} - ] Co-founding member, Johns Hopkins Data Science Lab
\item[\textnormal{2011} - ] Founding co-director, SMART research group
\item[\textnormal{2013} - ] Full professor, Department of Biostatistics, Johns Hopkins University
\item[\textnormal{2007-2013}] Associate professor, Department of Biostatistics, Johns Hopkins University
\item[\textnormal{2001-2007}] Assistant professor, Department of Biostatistics, Johns Hopkins University
\item[\textnormal{1996-1999}] Research assistant for professor Alan Agresti, Department of Statistics, University of Florida
\item[\textnormal{1996, 1999}] Intern /  database programmer, the Pediatric Oncology Group Statistical Office
\end{description}

\subsection*{Extended visits to other departments}
\begin{description}
\item[\textnormal{May - August 2006}] Department of Biostatistics, Emory University
\item[\textnormal{December - May 2006}] Center for Imaging Science, Johns Hopkins University
\item[\textnormal{June 2004}] Carnegie Mellon, Department of Statistics 
\end{description}

\section*{Professional activities}
\begin{description}{}{}
%\item Membership: ASA, ENAR, IMS, OHBM
\item Review of research proposals:
  \begin{description}
  \item NIH/NCI  2008; ad hoc study section member for {\it Quick Trials on Imaging and Image-guided Intervention}
  \item NIH/BMRD 2009; ad hoc study section member for {\it the Biostatistical Methods and Research Design Study Section}
  \item NIH/NIMH 2009 and 2010; ad hoc study section member for {\it Interventions Committee for Adult Disorders}
  \item NIH      2011; ad hoc study section member for {\it Special Emphasis Panel on Epidemiology} 
  \item NIH      2013; invited attendee, BD2K Workshop on Enhancing Training for Biomedical Data
  \item NIH		 2013 x2; ad hoc study section member for {\it Interventions Committee for Disorders Involving Children and Their Families}
  \item NIH 2014; ad hoc study section member for {\it Special Panel on Statistical Modeling}
  \item NIH/BD2K 2015; ad hoc study section member for {\it  BD2K Short Courses and Open Resource R25}
  \item NIH/NIMH 2016; ad hoc study section member for {\it Interventions/Biomarkers}.
  \item NIH/NIMH 2016; ad hoc study section member for {\it Research Education Programs}
    \item NIH/NIMH 2017; ad hoc study section member for {\it Interventions/Biomarkers}
  \item NIH 2017; ad hoc study section member for {\it Neural Basis of Psychopathology, Addictions and Sleep Disorders}
  \item NIH 2018; ad hoc study section member of {\it NeuroNEXT 2}
  \item NIH 2018; ad hoc study section member of {\it Healthcare Delivery and Methodologies}
  \end{description}
\item Professional society positions:
\begin{description}
\item Publications Officer for the Biometrics Section of the American Statistical Association 2005, 2006
\item Founding member (2010) and secretary (2010-) for the Statistics in Imaging ASA section.
\item Organizer: Biometrics invited session: Statistical Methodology for the Analysis of Sleep Studies, JSM 2007 
\item Organizer: Biometrics invited session: Statistical Methods for Complex Functional Biological Signals, ENAR 2011
\item Organizer: Contributed session: Novel developments in statistical blind source separation and independent components analysis, ENAR 2012
\item Session chair: JSM (2003, 2006, 2007), ENAR (2002, 2007), SAMSI (2013) 
\end{description}
\end{description}

\section*{Editorial activities}
\begin{description}
\item 2006-2008 Associate editor Computational Statistics and Data Analysis
\item 2008-2010 Associate editor for the Journal of the American Statistical Association
\item 2009-2012 Associate editor for the Journal of the Royal Statistical Society
Series B
\item 2010-2012 Associate editor for Biometrics
%\item Referee for: Australian Journal of Statistics, Biometrical
%  Journal, Biometrics, Biometrika, Biostatistics, Brain Imaging and
%  Behavior, Canadian Journal of Statistics, Circulation, Clinical
%  Trials, Computational Statistics, Computational Statistics and Data
%  Analysis, Computer Methods and Programs in Biomedicine, Electronic Journal of %Statistics, Handbook of
%  MCMC, IEEE Transactions on Medical Imaging, Information Sciences,
%  International Journal of Biostatistics, Journal of Computational and
%  Graphical Statistics, Journal of Neuroscience Methods, Journal of
%  Statistical Planning and Inference, Journal of Statistical Software
%  Development, Journal of the American Statistical Association,
%  Journal of the Royal Statistical Society, Statistics and its
%  Interface, Statistical Computing, Statistical Methods and
%  Applications, Statistical Modeling, Statistics and Probability
%  Letters, Statistics in Medicine
\item Book reviewer for: Springer-Verlag, Wiley
\item Human Brain Mapping 2009 abstract referee
\item Senior program committee member for the Fourteenth International Conference on Artificial Intelligence and Statistics 
\item Guest associate editor for Frontiers in Neuroscience special issue
on Brain Imaging Methods
\end{description}

\section*{Honors and awards}
\begin{description}
\item[\textnormal{1998}] William S. Mendenhall Award
\item[\textnormal{1999}] Anderson Scholar/Faculty nominee for the University of  Florida CLAS
\item[\textnormal{2001}] University of Florida CLAS Dissertation Fellowship
\item[\textnormal{2001}] University of Florida Statistics Faculty Award
\item[\textnormal{2002}] Johns Hopkins Faculty Innovation Award
\item[\textnormal{2006}] Johns Hopkins Bloomberg School of Public Health AMTRA award
\item[\textnormal{2008}] Johns Hopkins Bloomberg School of Public Health Golden Apple teaching award
\item[\textnormal{2011}] Leader and organizer of the declared winning entry of the 2011 ADHD200 prediction competition
\item[\textnormal{2011}] Presidential Early Career Award for
  Scientists and Engineers (PECASE, 2010, awarded in 2011); ``The highest honor bestowed by the
  United States government on science and engineering professionals in
  the early stages of their independent research careers''
\item[\textnormal{2013}] Organizer of the Kaggle/Heritage Health Prize team receiving 12th place out of 1,979 teams
\item[\textnormal{2014}] Named a Fellow of the American Statistical Association
\item[\textnormal{2015}] Special Invited Lecturer, European Meeting of Statisticians
\end{description}

%\bibliographystyle{plainyr-rev}
\bibliographystyle{cv}

\section*{Bibliography}
\begin{btSect}{pubs}
\subsection*{Research Articles}
%* - Advisees, co-advisees, post docs, K-grant advisees or mentored student research
\btPrintNotCited
\end{btSect}

\begin{btSect}{chapters}
\subsection*{Book chapters and encyclopedia entries}
\btPrintNotCited
\end{btSect}

\begin{btSect}{letters}
\subsection*{Letters and Editorials}
\btPrintNotCited
\end{btSect}

%\begin{btSect}{proceedings}
%\subsection*{Conference proceedings}
%\btPrintNotCited
%\end{btSect}

\begin{btSect}{other}
\subsection*{Other publications}
\btPrintNotCited
\end{btSect}

%\begin{btSect}{submitted}
%\subsection*{Manuscripts in preparation}
%\btPrintNotCited
%\end{btSect}

\section*{Software}
Software and subroutines relevant to my research can be downloaded at 
 \begin{verbatim}
http://www.bcaffo.com
http://www.smart-stats.org
\end{verbatim}
The software \texttt{exactLoglinTest} is listed at the Comprehensive R Archive Network. \\
github repository is at \texttt{github.com/bcaffo}.

%\section*{Media coverage and contributions}
%\begin{description}
%\item Data Science Series for Coursera covered in the Baltimore Sun, Washington Post and other %outlets
%\item Author of the BLOG post in the Huffington Post "WANTED: Neuro-quants" %http://huff.to/19gpqNH
%%\item Quoted in the 2013 Wall Street Journal "The Upbeat Stats on Statistics"
%\item Covered in the 2012 Washington Post article on MOOCs 
%"Elite education for the masses"
%\item The 2010 JAMA article was featured on Reuters, US
%  News, Business Week and other media outlets.
%\item Interviewed on the Dan Rodricks show (WYPR Baltimore) 1/25/2010
%\item Featured in the December 2009 issue of the Johns Hopkins School of Public Health %Magazine
%\item The 2009 PLOS Medicine manuscript was covered by: the Baltimore
%  Sun, CBS, AJC, US News and World Report, Bloomberg, Reuters and
%  other media outlets
%\item The 2008 Chest manuscript was covered by: the Washington Post,
%  Boston Globe, 7 News NBC Boston, the Toronto Star and other media
%  outlets
%\end{description}


\begin{center}
\large
\bf Curriculum Vitae\\
 Brian S. Caffo \\ 
 Part II
\end{center}

\section*{Teaching}
\subsection*{Advisees}
\begin{description}
\item[\textnormal{2005 PhD}] Leena Choi, Johns Hopkins Biostatistics, {\it Modelling biomedical data and the foundations of 
bioequivalence}; assistant professor, Vanderbilt University, Department of Biostatistics
\item[\textnormal{2006 ScM}] Lijuan Deng, Johns Hopkins Biostatistics, {\it Spline-based curve fitting with applications to 
kinetic imaging}; researcher, Harvard University
\item[\textnormal{2006 MS}] Bruce Swihart, University of Colorado
  Biostatistics, {\it Quantitative characterization of sleep
    architecture using multi- state and log-linear models} (jointly
  advised with Naresh Punjabi and Gary Grunwald); PhD student, Johns Hopkins Department of Biostatistics
\item[\textnormal{2007 MPH}] Jeong Yun, Johns Hopkins Bloomberg School of Public Health, {\it Incidence of hypertension in high risk groups of the Sleep Heart Health Study}
\item[\textnormal{2008 PhD}]  Xianbin Li, Johns Hopkins Biostatistics, {\it Modeling composite outcomes and their 
component parts}; researcher, US Food and Drug Administration
\item[\textnormal{2008 PhD}] Shu-Chih Su, Johns Hopkins Biostatistics, {\it Structure/function relationships in the analysis 
of anatomical and functional neuroimaging data}; researcher, Merck Pharmaceuticals
\item[\textnormal{2010 ScM}] John Muschelli, Johns Hopkins Biostatistics, {\it An iterative approach to hemodynamic response function temporal derivatives in statistical parametric mapping for functional neuroimaging}; PhD student, Johns Hopkins Department of Biostatistics
\item[\textnormal{2011 PhD}] Haley Hedlin, Johns Hopkins Biostatistics, {\it Statistical methods for inter-subject analysis of neuroscience data}; post doctoral student, Department of Mathematics and Statistics, University of Massachusetts
\item[\textnormal{2011 PhD}] Bruce Swihart, Johns Hopkins Biostatistics, {\it From individuals to populations:
application and insights concerning the generalized linear mixed model}; post doctoral student, Johns Hopkins University Department of Biostatistics
\item[\textnormal{2012 PhD}] Jeff Goldsmith (co-advised with primary advisor Ciprian Crainiceanu), {\it Cross-Sectional and longitudinal penalized functional regression}; assistant professor, Department of Biostatistics, Columbia University
\item[\textnormal{2012 MPH}] Tiziano Marovino, {\it The concurrent validity of musculo-skeletal ultrasound imaging in comparison to
MRI for detecting rotator cuff tears in the shoulder when performed in a physical therapy setting}
\item[\textnormal{2013 ScM}] Rawan Al-Lozi, {\it An evaluation of statistical modeling
methods for predicting recovery time from post-traumatic amnesia following moderate or severe traumatic brain injury in children}.
\item[\textnormal{2013 PhD}] Shanshan Li (co-advised with primary advisor Mei-Cheng Wang)
 {\it Statistical Methods for Evaluating Diagnostic Accuracy of Biomarkers}; assistant professor Indiana University-Purdue University  Indianapolis Biostatistics.
\item[\textnormal{2013 MHS}] Xiaoqiang Xu, {\it Parallel Voxel Level Anything}
\item[\textnormal{2015 PhD}] Juemin Yang {\it Statistical Methods for Brain
Imaging and Genomic Data Analysis}; researcher Citibank
\item[\textnormal{2015 PhD}] Shaojie Chen {\it Statistical Methods to Analyze Massive High-Dimensional Neuroimaging Data.}
\item[\textnormal{2015 PhD}] Fang Han (Co-advised with Han Liu) {\it Large-scale nonparametric and semiparametric inference for large complex and noisy datasets}
\item[\textnormal{2016 PhD}] Chen Yue (co-advised with Vadim Zipunnikov) {\it Generalizations, extensions and applications for principal component analysis.}
\item[\textnormal{2016 PhD}] Amanda Mejia (co-advised with primary advisor Martin Lindquist) {\it Statistical Methods for Functional Magnetic Resonance Imaging Data. }
\item[\textnormal{2016 PhD}] Aaron Fisher {\it Methods for High Dimensional Analysis, Multiple Testing, and Visual Exploration }
\item[\textnormal{2016 PhD}] Huitong Qiu {\it Statistical Methods and Theory for  Analyzing High Dimensional Time Series}.
\item[\textnormal{2020 PhD}] Zeyi Wang {\it Statistical Analysis of Functional Connectivity in Brain Imaging:
Measurement Reliability and Clinical Applications}
\item[\textnormal{2020 MSE}] Luchao Qi {\it Associations between Body Mass Index (BMI) and Physical
Activity: National Health and Nutritional Examination Survey (NHANES) 2005-2006}
\end{description}

\subsection*{Postdoctoral advisees}
\begin{description}
\item[\textnormal{2009-2012}] Vadim Zipunnikov (co-advising with primary advisor Ciprian Crainiceanu)
\item[\textnormal{2010-2013}] Ani Eloyan (co-advising with Ciprian Crainiceanu)
\item[\textnormal{2011-2013}] Seonjoo Lee (co-advising with primary advisor Dzung Pham)
\item[\textnormal{2017-2020}] Yi Zhao (co-advising with Stewart Mostofsky and Martin Lindquist)
\item[\textnormal{2017-2020}] Heather Shappell (co-advising with Jim Pekar and primary advisor Martin Lindquist)
\end{description}

\subsection*{Advisees in progress}
\begin{description}
\item Huan Chen
\item Bohao Tang
\item Bingkai Wang (Primary advisor Michael Rosenblum)
\end{description}

\subsection*{Interns}
\begin{description}
\item[\textnormal{2013}] Nick Carchedi, Ethan Schwartz, Lauren Williams
\item[\textnormal{2010}] Katie Phelan
\end{description}

%\subsection*{K award mentees}
%Ying Cao, Madhav Goyal, Daniel Harrison, Abhinav Kumar Jha, 
%Saman Nazarian, Sheryl Rimrodt, Adam Spira, Stacy Suskauer

\subsection*{Academic advisees}
\begin{description}
\item[\textnormal{Doctoral students}] Xianbin Li, Yun Lu, Huitong Qiu 
\item[\textnormal{ScM students}]  Lijuan Deng  
\item[\textnormal{MHS students}]  Nan Guo, Juleen Lam, Fengmin Zhao, Jiemin Ma, Carolyn Scrafford
\item[\textnormal{MPH students}]  Hana Lee, Tiziano Marovino, Sri-sujanthy Rajaram, Elizabeth Wheler, Jeong Yun 
\end{description}

\subsection*{Master's thesis reader} 
\begin{description}
\item[\textnormal{2014}] Clair Rock (U MD Epi)
\item[\textnormal{2012}] Rawan Al-Lozi, Francisco Leva
\item[\textnormal{2011}] Jiawei Bai (Biostat), Pohan Chen (Biostat)
\item[\textnormal{2010}] Ben Althouse (Biostat) 
\item[\textnormal{2009}] Catherine Thomas (Biostat), Ros Reside (Epi) 
\item[\textnormal{2006}] Ricardo Carvalho (GTPCI), Bruce Swihart (UC Denver Biostatistics) 
\item[\textnormal{2005}] Brendan Click (Biostat), Jennifer Ryea (Biostat) 
\item[\textnormal{2004}] Meh Fen Yeh (Biostat) 
\end{description}

%\subsection*{Notable awards won by former advisees and coadvisees during their training}
%\begin{description}
%\item[\textnormal{Merrell Award:}] Jeff Goldsmith, Shanshan Li, Fang Han, Amanda Mejia, Aaron Fisher
%\item[\textnormal{GlaxoSmithKline Award:}] Shanshan Li
%\item[\textnormal{June B Culley Award:}] Juemin Yang, Shanshan Li, Haley Hedlin and Bruce Swihart
%\item[\textnormal{Helen Abbey Award:}]  Jeff Goldsmith and Bruce Swihart
%\item[\textnormal{The Louis I and Thomas D Dublin Award:}] Bruce Swihart
%\item[\textnormal{The Jane and Steve Dykacz Award:}] Jeff Goldsmith
%\item[\textnormal{Distinguished Student Paper Award, ENAR:}] Haley Hedlin, Jeff Goldsmith
%\end{description}


\subsection*{Preliminary oral participation}
\begin{description}
\item[\textnormal{2020}] Huan Chen (Biostat), Bohao Tang (Biostat), Vy Tran (EHE), Guoqing Wang (Biostat)
\item[\textnormal{2019}] Vikram Chandrashekhar (BME, GBO), Lacey Etzkorn (Biostat)
\item[\textnormal{2018}] Joshua Cappe (AMS), Celia Carp (PFRH), Bingkai Wang (Biostat), Zeyi Wang (Biostat)
\item[\textnormal{2017}] Runze Tang (AMS)
\item[\textnormal{2016}] Eugenie Shieh (GTPCI)
\item[\textnormal{2015}] John Muschelli (Biostat)
\item[\textnormal{2014}] Hiwot Hiruy (GTPCI)
\item[\textnormal{2014}] Shaojie Chen (Biostat), Aaron Fisher (Biostat), Huitong Qiu (Biostat) 
\item[\textnormal{2013}] Fang Han (Biostat), Sarah Khasawinah, Gwenyth Lee (IH), Jenna Riis (PFRP), Chen Yue (Biostat)
\item[\textnormal{2012}] Andrew Pike (MMI), Tom Prior (Biostat), Haochang Shou (Biostat), Zhenke Wu (Biostat), Juemin Yang (Biostat)
\item[\textnormal{2011}] Melania Bembea (GTPCI), Yifang Chuang (MH), Jenna Krall (Biostat), Shanshn Li (Biostat), Saman Nazarian (Epi), Adrienne Tin (Epi)
\item[\textnormal{2010}] Bradley Foerster (GTPCI), Jeff Goldsmith (Biostat), Attia Goheer (Epi), Xiaoxu Kang (BME), Maggie Kuo (BME), Yan Ning (Biostat),  Carolyn Scrafford (IH), Yajing Yang (BME)
\item[\textnormal{2009}] Vikram Aggarwal (BME)
\item[\textnormal{2008}] Soumyadipta Acharya (BME), Haley Hedlin (Biostat), Alan Huang (BME), Yang Hui (HPM) , 
Jun Hua (EE),  Zhiliang Ma (AMS),  Gila Neta (Epi), Adam Stakenas (AMS), Bruce Swihart (Biostat), James Williams (MH),  
\item[\textnormal{2007}] Gabriel Lai (Epi), Issel Lim (BME), Greta Mok (EHS), Erin Rand-Giovanetti (HPM), Hilary Schwandt (PFH),  Kenneth Shermock (HPM), Stella Yi (Epi) 
\item[\textnormal{2006}] Ying Cao (GTPCI), Yu-Jen Chen (Biostat), Alison Laffan (Epi), Taek Soo Lee (EHS), Xianbin 
Li (Biostat), Shu-Chih Su (Biostat) 
\item[\textnormal{2005}] Leslie Cromwell (HPM), Bin He (EHS) 
\item[\textnormal{2004}] Kenneth Brenneman (EHS), Elizabeth Johnson (Biostat), Rongheng Lin (Biostat)
\item[\textnormal{2003}] Yi Huang (Biostat), Lin Zhang (Epi) 
\item[\textnormal{2002}] Dongmei Liu (Biostat), Samuel Mills (PFH)  
\end{description}


\subsection*{Final oral participation}
\begin{description}
\item[\textnormal{2020}] Michelle Hawks Cuellar (PFRH), Celia Karp (PFRH),  Zeyi Wang (Biostat), 
\item[\textnormal{2019}]  Courtenay Holscher (GTPCI), Eugenie Shieh (GTPCI), Ethel Weld (GTPCI)
\item[\textnormal{2018}] Roxanne Mirabal-Beltran (PFRH)
\item[\textnormal{2016}] Yue Chen (Biostat), Aaron Fisher (Biostat), Amanda Mejia (Biostat), John Muschelli (Biostat), Huitong Qiu (Biostat)
\item[\textnormal{2015}] Hiwot Hiruy (GTPCI)
\item[\textnormal{2014}] Shaojie Chen (Biostat), Juemin Yang (Biostat)
\item[\textnormal{2014}] Sarah Kashwinah (MH), Jenna Krall (Biostat), Haochang Shou (Biostat), Zhenke Wu (Biostat)
\item[\textnormal{2013}] Bradley Forster (GTPCI), Shanshan Li (Biostat), Carolyn Scrafford (IH)
\item[\textnormal{2012}] Yifang Chang (MH), Nan Guo (Epi), Jing Hua (Epi), Gwenyth Lee (IH), Allan Massie (Epi), Caroline Min (PFHS),  Saman Nazarian (Epi)
\item[\textnormal{2011}] Haley Hedlin (Biostat), Jennifer Schrack (Epi), Jennifer Feder (Biostat), Bruce Swihart (Biostat)
\item[\textnormal{2009}] Kenneth Brenneman (EHS), Greta Mok (EHS), Alison Mondul (Epi) James R Williams (MH) 
\item[\textnormal{2008}] Ying Cao (GTPCI), Ingrid Frieberg (IH), Alison Laffan (Epi), Xianbin Li (Biostat), Chi Liu 
(EHS), Shu-Chih Su (Biostat) 
\item[\textnormal{2007}] Leslie Conwell (HPM), Yue Yin (Biostat) 
\item[\textnormal{2006}] Hongfei Guo (Biostat), Bin He (EHS), Bruce Swihart (UC Denver Master’s thesis defense) 
\item[\textnormal{2005}] Leena Choi (Biostat), Mike Griswold (Biostat), Dongmei Liu (Biostat), John Majnu (AMS), 
Susan Milner (2005) 
\item[\textnormal{2004}] Samuel Mills (PFH), Judy Ng (HPM), Lin Zhang (Epi)
\end{description}

\subsection*{Classroom Instruction}
\subsubsection*{Johns Hopkins} 
\begin{description}
\item[\textnormal{2001-2005}] Primary instructor, Advanced Statistical Computing  Biostatistics PhD elective 10-20 students 
\item[\textnormal{2003-2004}] Primary instructor, Advanced Methods in Biostatistics IV Biostatistics PhD and ScM core requirement 10-20 students 
\item[\textnormal{2003-2004}] Guest lecturer, Advanced Methods in Biostatistics II
Biostatistics PhD and ScM core requirement (Two weeks of lectures on linear mixed models) 10-20 students 
\item[\textnormal{2003-2008}] Guest lecturer, Computing orientation and student computing club 
\item[\textnormal{2003-2004, 2008}] Lead instructor, Statistical Computing Biostatistics elective 20-30 students
\item[\textnormal{2004-2005}] Primary instructor, Advanced Methods in Biostatistics III 
Biostatistics PhD and ScM core requirement 20 students 
\item[\textnormal{2005-2010}] Primary instructor, Methods in Biostatistics  I 
Biostatistics PhD, ScM core requirement  60 students 
\item[\textnormal{2005-2010}] Primary instructor, Methods in Biostatistics II
  Biostatistics PhD, ScM core requirement 60 students
\item[\textnormal{2008}]  Primary instructor, Medical Imaging Statistics, Biostatistics PhD and ScM elective lectures 10 students 
\item[\textnormal{2010-2017}] Guest lecturer, Public Health Perspectives Biostatistics Module
\item[\textnormal{2011-2014}] Primary instructor, Advanced Methods in Biostatistics I and II,
Biostatistics PhD requirement 15 students
\item[\textnormal{2013}] Guest instructor, ICTR training program
\item[\textnormal{2015-2018}] Primary instructor, Advanced Linear Models I and II
\item[\textnormal{2019}] Primary instructor, Data Science for Biomedical Engineering
\end{description}

\subsection*{Open Education}
\subsubsection*{Coursera}
\begin{description}
\item Mathematical Biostatistics Boot Camp - 7 week course 
\item Mathematical Biostatistics Boot Camp 2 - 7 week courses 
\item Advanced Linear Models for Data Science 1: Least Squares
\item Advanced Linear Models for Data Science 2: Statistical Linear Models
\item Data Specialization (with Roger Peng and Jeff Leek); 9 one month classes run monthly plus a  capstone project class; primary instructor for:
\begin{description}
\item Statistical Inference
\item Regression Models
\item Developing Data Products
\end{description}
\item Executive Data Science Specialization (with Roger Peng and Jeff Leek); 4 one month classes plus a capstone project
\item PI (roll of executive producer, non-instructor) for the BD2K R25 Genomic Data Science Specialization, fMRI 1 and 2 (Lindquist / Wager), Neurohacking in R (Craininceanu, Sweeney, Muschelli), Neuroscience for Neuroimaging (Baker)
\end{description}

\subsubsection*{Leanpub (e-books)}
\begin{description}
    \item Statistical Inference
    \item Regression Models
    \item Developing Data Products
    \item Advanced Linear Models for Data Science
    \item Methods in Biostatistics with R (with John Muschelli and Ciprian Crainiceanu)
    \item Executive Data Science (with Roger Peng and Jeff Leek)
\end{description}

\subsubsection*{Short courses and hackathons}
\begin{description}
\item[\textnormal{2011 ENAR}] Statistical methods for new high throughput technological measurements; with Ciprian Crainiceanu
\item[\textnormal{2015 Data Science Hackathon}] Co-organizer with
Leah Jager, Jeff Leek and Roger Peng. Funded by the NIH
\item[\textnormal{2017-2019 quarterly MRICloud and R}] Johns Hopkins tutorial series
\end{description}


\subsubsection*{Other} 
\begin{description}
\item swirl: Mentored project by Nick Carchedi intiated during his internship
\item Infinity University: leader and executive producer; in progress early childhood data science education; co-lead with Jessica Crowell (director) and Michael Orzekowski (script writer, voice talent)
\item Course notes for Biostatistics 140.651-2 listed on the Johns Hopkins Open Courseware project 
%\item Experience tutoring and teaching students with mathematics learning disabilities  
\item Primary instructor, TA, and tutor for introductory and intermediate statistics and remedial mathematics 
courses at the University of Florida; primary instructor course enrollments ranged from 20 to 400 
students 
\item Created a statistics course for the McNair’s Scholar program, a minority recruitment and retention 
program at the University of Florida. 
\end{description}

\section*{Research support} 
\subsection*{Principal investigator or Co-PI}
\begin{description}
\item[\textnormal{8/27/12 - 8/26/14}] Johns Hopkins Brain Science Institute {\it The Center for Quantitative Neuroscience: a core for population neuroanalytics and translational systems neuroscience} as part of the RFA {\it Traumatic brain injury: mechanisms and treatment}.
\item[\textnormal{09/30/2010-8/31/2014}] NIH NIBIB R01 EB012547 {\it Statistical methods for hierarchical large n large p problems} Modern observational data is often longitudinal or multilevel functional biological signals. We propose a foundational approach for the analysis of such data, including scalable computing to next-generation data sets.
\item[\textnormal{05/01/06-04/30/09}] NIH NIBIB K25 EB003491 {\it A
    mentored training program in imaging science} 
  The aims of this proposal are to accelerate EM based iterative
  reconstruction algorithms and to theoretically and empirically
  investigate intra-iteration smoothing.  All of the developed
  algorithms will be extensively tested using Monte Carlo and actual
  patient data.
\item[\textnormal{12/01/14-11/30/17}] NIH R25EB020378 {\it Big Data Education for the Masses: MOOCs, Modules and Intelligent Tutoring Systems} We propose two Massive Open Online Course series in neuroimaging and genomic Big Data analysis as well
as the creation of modular Big Data statistics content and content creation for an intelligent tutoring system.
\item[\textnormal{09/01/11- 08/31/16}] NIH NIBIB P41 EB015909 {\it Resource for quantitative functional MRI} R01 component of a P41 grant. R01 PIs Caffo/Pekar, P41 PI Van Zijl. The work in this subaim will consider research in statistical models for the analysis of functional MRI-based connectivity.
\item[\textnormal{3/14/2012-3/14/2014}] Amazon AWS Research Grant for Cloud Development of Neuroimaging Software.
\end{description}

\subsection*{Co-investigator} Available by request.

\section*{Academic service} 
\subsection*{Major committee involvement}
\begin{description}
\item JHSPH Faculty senate president-elect 2019-2020
\item JHSPH Honors and Awards committee 2017-2019
\item JHSPH Faculty Innovation Fund grant referee 2017 - 2018
\item BME faculty hiring committee 2016
\item Biostat faculty hiring committee 2016
\item Malone Center Steering Committee 2016 -
\item Biostatistics admissions committee member 2002 - 2009, 2010 -
\item Biostatistics co-director of the graduate program 2010 - 2012
\item Biostatistics director of the graduate program 2012 - 2016
\item Biostatistics information technology committee member 2001 - 2009 
\item Biostatistics seminar coordinator 2001 - 2002 
\item Co-director Biostatistics/Epidemiology MPH concentration 2008 - 2010
\item Co-organizer Junior Faculty Meetings 2003 
\item Committee on Affirmative Action member 2007 - 2010
\item Faculty Senate representative 2002 - 2004 
\item MPH Admissions Committee member 2009 - 2011
\item MPH Executive Board member 2009 -  2011
\end{description}

\subsection*{Example other service work}
\begin{description}
\item Biostatistics faculty representative to CEPH site visit 2006 
\item Biostatistics second year examination committee 2003-2005 
\item Biostatistics self study committee 2007 
\item Developmental Disabilities Task Force representative 2007-2009 
\item Member of cancer/epi search committee 2008 
\item Member of ad hoc committee to review faculty hiring for the Committee on Affirmative Action 2008 
\end{description}

\subsection*{Johns Hopkins statistical consulting}
\begin{description}
\item Leader of the IDRC biostatistics consulting core for Kennedy Krieger 
\item Member of the CTSA biostatistics consulting core  
\item Member of the DSMB for {\it Effect of n-CPAP Treatment on Glycemic Control  in patients with  Type 2 
Diabetes Mellitus and Obstructive Sleep Apnea GLYCOSA}
\end{description}

\subsection*{External statistical consulting}
\begin{description}
\item d8alab 2016, co-founder and consultant
\item Sapphire consulting, July 2008, October 2010, January 2011 
\item Creative Business Strategies International, July 2008
\item Pfizer Pharmaceuticals, one year research contract (PI Dr. Bruno Jedynak), 2011
\item Merck Pharmaceuticals, one year research contract (PI Dr. Ciprian Crainiceanu), 2011
\item AgeneBio, October 2011
\end{description}
\section*{Presentations}

\subsection*{Invited seminars or seminars with peer reviewed applications}
\begin{description}
    \item[\textnormal{2001}] 
    \begin{description}
        \item {\it ESUP accept/reject sampling}, North Carolina State University Department of         Statistics, Raleigh, North Carolina.
        \item {\it Monte Carlo exact conditional hypothesis tests for loglinear models}, AT\&T Labs, Florham Park, New Jersey. 
        \item  {\it Monte Carlo exact conditional hypothesis tests for loglinear models}, Fifth Workshop on Groebner Bases and Statistics (GROSTAT V), Tulane University, New Orleans, Louisiana.
        \item {\it Monte Carlo exact conditional hypothesis tests for loglinear models}, Johns Hopkins University Department of Biostatistics, Baltimore, Maryland.
        \item {\it Monte Carlo exact conditional hypothesis tests for loglinear models}, University of Michigan Department of Statistics, Ann Arbor, Michigan.
        \item {\it Monte Carlo exact conditional hypothesis tests for loglinear models}, Ohio State University Department of Statistics, Columbus, Ohio.
    \end{description}
    \item[\textnormal{2002}] 
    \begin{description} 
        \item {\it Model selection and fitting for empirical Bayes analysis of microarray data}, Joint Statistical Meetings New York, New York.
        \item {\it Ascent-based MCEM}, Yale University Division of Biostatistics, New Haven, Connecticut.
        \item {\it ESUP accept/reject sampling}, Johns Hopkins University Department of Biostatistics, Baltimore Maryland.
    \end{description}
    \item[\textnormal{2003}] 
    \begin{description}
        \item {\it A tour of biostatistics}, Drexel University Department of Mathematics, Philadelphia, Pennsylvania.
        \item {\it ESUP accept/reject sampling}, Duke University Institute of Statistics and Decision Sciences, Durham, North Carolina .
        \item {\it Missing data and air pollution}, Drexel University Department of Mathematics, Philadelphia, Pennsylvania.
        \item {\it Monte Carlo conditional analysis for loglinear and logistic models}, Joint Statistical Meetings, San Francisco, California.
        \item {\it Monte Carlo conditional analysis for loglinear and logistic models}, Statistics and Applied Mathematical Sciences Institute, Workshop on Exact Categorical Methods, Research Triangle Park, North Carolina
    \end{description}
    \item[\textnormal{2004}]
    \begin{description}
        \item {\it Multilevel models with applications in genomics}, University of Minnesota Department of Statistics, Minneapolis, Minnesota.
        \item {\it Ascent-based MCEM}, Cornell University Department of Statistics, Ithaca, New York.
    \end{description}
    \item[\textnormal{2005}]
    \begin{description}
        \item {\it Ascent-based MCEM}, Johns Hopkins University Department of Applied Math and Statistics, Baltimore, Maryland.
        \item {\it ESUP accept/reject sampling}, Pennsylvania State Department of statistics, University, College Station, Pennsylvania.
        \item {\it A tutorial on statistical power calculations}, Johns Hopkins University Center for Mind Bind Research, Baltimore, Maryland.
        \item {\it Discussion of: characterizing experimentally induced neuronal processing by DuBois Bowman}, Department of Biostatistics Grand Rounds, Johns Hopkins University, Department of Biostatistics.
        \item {\it Quantitative characterization of chloroquine and aspirin in the male genital tract}, with Craig Hendrix, Johns Hopkins Division of Clinical Pharmacology, Baltimore, Maryland.
    \end{description}
    \item[\textnormal{2006}]
    \begin{description}
        \item {\it Ascent-based MCEM}, Department of Statistics, Carnegie Mellon University, Pittsburgh, Pennsylvania.
        \item {\it Is MRI based structure a mediator for lead’s effect on cognitive function}, MICE meeting, Welch Center for Prevention, Epidemiology and Clinical Research, Baltimore, Maryland.
    \end{description}
    \item[\textnormal{2007}]
    \begin{description}
        \item {\it A Bayesian hierarchical framework for spatial modeling of fMRI data}, Center for Statistics in the Social Sciences, University of Washington, Seattle, Washington.
        \item {\it A case study in pharmacologic imaging using single photon emission computed tomography}, UMBC Prob/Stat Day, Baltimore, Maryland.
        \item {\it Age, lead exposure and neuronal volume}, ENAR, Atlanta, Georgia.
        \item {\it Generalized linear mixed model analysis of multistate sleep transition data: the Sleep Heart Health Study}, Joint Statistical Meetings, Salt Lake City, Utah.
        \item {\it Statistical methods for indirect estimation of physiological parameters: case studies in viral kinetics}, Department of Statistics University of Minnesota, Minneapolis, Minnesota.
        \item {\it Statistical methods in functional medical imaging}, Department of Biostatistics, University of Florida, Gainesville, Florida.
    \end{description}
    \item[\textnormal{2008}]
    \begin{description}
        \item {\it A Bayesian hierarchical framework for spatial modeling of fMRI}, Human Brain Mapping, Melbourne, Australia.
        \item {\it Conditional and marginal models for binary outcomes}, Department of Statistics University of Minnesota, Minneapolis, Minnesota.
        \item {\it Lead exposure, neuronal volume and cognitive function}, Department of Biostatistics University of Florida, Gainesville, Florida.
        \item {\it Non-linear curve fitting in the analysis of medical imaging data}, Department of Biostatistics Grand Rounds, Johns Hopkins University, Baltimore, Maryland.
        \item {\it Pharmacologic imaging using principal curves in single photon emission computed tomography}, ENAR, Arlington, Virginia.
        \item {\it Quantifying the hypnogram and sleep stage transitions: novel approaches and applications to sleep disorders}, Annual Meeting of the Associated Professional Sleep Societies, Baltimore, Maryland.
        \item {\it Statistical methods for indirect estimation of physiological parameters: case studies in viral kinetics}, Department of Biostatistics, Columbia University, New York, New York.
        \item {\it Statistical methods for indirect estimation of physiological parameters: case studies in viral kinetics}, Department of Biostatistics, Emory University, Atlanta, Georgia.
        \item {\it Statistical methods for indirect estimation of physiological parameters: case studies in viral kinetics}, Department of Biostatistics, Vanderbilt University, Nashville, Tennessee.
    \end{description}
    \item[\textnormal{2009}]
    \begin{description}
        \item {\it Non-linear curve fitting in the analysis of medical imaging data}, Center for Imaging Science, Department of Biomedical Engineering, Johns Hopkins University, Baltimore, Maryland.
  \item {\it Non-linear curve fitting in the analysis of medical
      imaging data}, University of Pittsburgh, Department of
    Biostatistics, Pittsburgh, Pennsylvania.
  \item {\it Non-linear regression, an overview}, Statistics Without
    the Agonizing Pain Series, Johns Hopkins University, Baltimore,
    Maryland.
    \item {\it On the analysis of multiple sleep hypnograms},
      International Statistical Institute, Durban, South Africa.
    \item {\it Statistical methods for studying connectivity in the
        human brain}, International Workshop on Statistical Modeling,
      Ithaca, New York.
    \end{description}
\item[\textnormal{2010}]
  \begin{description}
	\item {\it Functional principal components for high dimensional brain volumetrics}, International Workshop on Statistical Modeling, Glasgow, Scotland.
    \item {\it Statistical methods for evaluating connectivity in the human brain}, ENAR, New Orleans, Louisiana. 
    \item {\it Statistical methods for high dimensional imaging studies of populations}, Department of Psychiatry and Behavioral Science, 
Johns Hopkins Bayview Medical Center, Baltimore, Maryland. 
  \end{description}
\item[\textnormal{2011}]
    \begin{description}
	\item {\it fMRI functional connectivity in subjects at high familial risk for Alzheimer's disease: new approaches to analysis}, Dementia Consortium, Johns Hopkins, Baltimore, Maryland
    \item {\it Indirect estimation of kinetic parameters in dual isotope single photon emission computed tomography studies of  microbicide lubricants}, ENAR, Miami, Florida.

    \item {\it Statistical methods for studying connectivity in the
        human brain}, Division of Biostatistics, University of Maryland, Baltimore, Maryland.
    \item {\it Statistical methods for studying connectivity in the
        human brain}, Department of Biostatistics, University of Washington, Seattle, Washington.
    \item {\it Statistical methods for studying connectivity in the
        human brain}, Department of Statistics, Cornell University, Ithaca, New York. 	
     \item {\it Statistical methods for studying connectivity in the
        human brain}, Dementia Consortium, Johns Hopkins, Baltimore, MD.
    \item {\it An overview of EEG research at Hopkins Biostatistics}, Regional EEG/ERP Conference, 
      Kennedy Krieger Institute, Baltimore, MD.
    \item {\it Statistical methods for evaluating human brain connectivity}, Statistical Methods for Very Large Data Sets Conference, Baltimore, MD.
    \item {\it Statistical methods for evaluating human brain connectivity}, The Brad Efron Honorary Symposium on Large-Scale Inference, Silver Springs, MD.
    \item {\it Statistical methods for evaluating human brain connectivity}, ISDS, Duke University, Durham, NC.
\end{description}
\item[\textnormal{2012}]
    \begin{description}
      \item {\it Predicting neurological disorders using functional and structural brain imaging data}, ENAR, Washington DC.
\item {\it Predicting neurological disorders using functional and structural brain imaging data}, Department of Statistics, University of Virginia, Charlottesville, Va. 
\item {\it Panelist at the 2012 NIH/NIBIB training grantee meeting}, National Institutes of Health, Bethesda, MD.
\item {\it Resting state brain functional connectivity: progress, future challenges and data},
 SAMSI opening workshop on massive data, Raleigh, NC.
\item {\it Statistical analysis of functional MRI resting state brain connectivity data}, Departments of Statistics and Biostatistics, University of Wisconsin, Madison, Wisconsin
\item {\it Statistical analysis of functional MRI resting state brain connectivity data}, Departments of Biostatistics, Yale University, New Haven, Connecticut.
    \end{description}
\item[\textnormal{2013}]
  \begin{description}
  \item {\it Large scale decompositions for functional imaging studies}, ENAR, Orlando, Florida.
    \item {\it Homotopic group ICA for resting state fMRI}, SAMSI NDA workshop, Raleigh, North Carolina.
    \item {\it Analyzing neurological disorders using functional and structural brain imaging data}, Department of Child and Adolescent Psychiatry, NYU, New York, New York.
    \item {\it Analyzing neurological disorders using functional and structural brain imaging data}, Department of Statistics, Virginia Tech, Blacksburg Virginia.
    \item {\it Analyzing neurological disorders using functional and structural brain imaging data}, Microsoft Research, Redmond, Washington.
  \end{description}
\item[\textnormal{2014}]
    \begin{description}
    \item {\it Teaching statistics for the future: the MOOC revolution and beyond}, Division of Biostatistics, Washington University, Saint Louis, Missouri
    \item {\it Developmental Disorders and Neuroimaging: Tools, Results and Issues}, ENAR, Baltimore, MD
    \item {\it Teaching statistics for the future: the MOOC revolution and beyond}, Department of Epi and Biostat Grand Rounds, U of MD, Baltimore, MD
    \item {\it Teaching statistics for the future: the MOOC revolution and beyond}, Dean's Lecture, Bloomberg School of Public Health, Baltimore, MD
    \item {\it Analyzing neurological disorders using functional and structural brain imaging data}, ISBIS SLDM joint meetings, Durham, NC
    \item {\it Statistical methods for the study of human brain functional connectivity}, JSM, Boston, MA
    \item Panelists for {\it Great Expectations: Training Future Biostatisticians for Careers in Interdisciplinary Biomedical Research}, JSM, Boston MA
	\end{description}
\item[\textnormal{2015}]     
    \begin{description}
	\item {\it Analyzing neurological disorders using functional and structural brain imaging data}, University of Pennsylvania, Philadelphia, PA
	\item {\it Discussion of: Statistical Quantitative Magnetic Resonance Imaging by Dr Taki Shinohara}, Johns Hopkins Department of Biostatistics, Baltimore, MD
	\end{description}
\item[\textnormal{2016}]
    \begin{description}
        \item {\it Bar Codes, Fingerprints and Reproducibility in Functional and Structural Brain Imaging Data }, Maryland Neuroimaging Retreat, Baltimore, MD
        \item {\it Hypothesis Driven Research}, Society for Neuroscience Webinar
        \item {\it EDA}, NIH BD2K Training Program Webinar
    \end{description}
\item[\textnormal{2017}] 
    \begin{description}
        \item {\it Bar Codes, Fingerprints and Reproducibility in Functional and Structural Brain Imaging Data }, Department of Biostatistics, McGill University, McGill, Canada.
    \end{description}
\item[\textnormal{2018}]
    \begin{description}
        \item {\it Bar Codes, Fingerprints and Reproducibility in Functional and Structural Brain Imaging Data}, Department of Biomedical Engineering, Johns Hopkins University, Baltimore, Maryland.
        \item {\it Will the doctor of the future be a human, robot or cyborg} Mayo Clinic, Rochester, Minnesota. 
        \item {\it Overcoming statistical paralysis} Society for Neuroscience webinar.
        \item {\it The future of data science education}, Keynote talk for the STEM Powered Education conference at the University of Florida.
        \item {\it The future of data science education}, Becton Dickinson, Franklin Lakes, New Jersey.
        \item {\it Deep learning in public health and personalized medicine} Johns Hopkins Bloomberg School of Public Health, Baltimore, MD.
    \end{description}
\end{description}
\end{document}

 \section*{keywords} 
Medical imaging, MRI, DTI, PET, SPECT, expectation maximization, generalized linear mixed models, log- 
linear models, magnetic resonance imaging, mixed models, multi-level models 



\section*{Additional information} 
~~~~My research involves computationally intensive statistical methods. I apply these methods to models for 
clustered and longitudinal categorical data and medical image processing and analysis. My current research 
agenda focuses on the analysis of experimental medical imaging data.  
 
~~~~A specific focus of my research is the computation and application of statistical tools called multilevel models.  
This flexible class of statistical techniques, which includes linear mixed effect models, generalized- and non- 
linear mixed effect models and Bayesian models, are applicable to a nearly endless variety of scientific settings.  
In particular, I have applied these methods to social science data, general contingency table data, 
pharmacological data, brain imaging region of interest studies, voxel-by-voxel brain imaging studies, 
electroencephalogram data and sleep transition data. 
 
~~~~Of late, I have endeavored to focus my research on the application and 
computation of multilevel models to medical imaging data.  I have focused on nuclear medicine and brain 
imaging studies from modalities such as: positron emission tomography, single photon emission computed 
tomography, magnetic resonance imaging, and functional magnetic resonance imaging.   
 
~~~~My more theoretical work focuses on the computational aspects of these models and general computing 
algorithms. As such, I have worked on likelihood based marginalized models for categorical data, the 
expectation/maximization algorithm and Markov chain Monte Carlo.  
